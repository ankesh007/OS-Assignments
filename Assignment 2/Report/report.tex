\documentclass{article}
\usepackage{graphicx}
\usepackage{geometry}
\usepackage{amsmath,mathpazo}
\usepackage{amssymb}
\usepackage{mathtools}
\usepackage{commath}
\usepackage{enumitem}
\usepackage{listings}
\usepackage{grffile}

\geometry{
  top=20mm,
}
\newcommand{\boldvec}[1]{\boldsymbol{\vec{\textbf{#1}}}}
\newcommand{\capvec}[1]{\boldsymbol{\hat{\textbf{#1}}}}
\newcommand{\bld}[1]{\textbf{#1}}
\newcommand{\ital}[1]{\textit{#1}}
\newcommand{\italb}[1]{\textbf{\textit{#1}}}
\begin{document}

\title{Operating Systems- COL331 \\Assignment 2}
\author{Ankesh Gupta\\2015CS10435}

\date{}
\maketitle

\section*{Part 1}
\bld{Pointers on printing currently running processes}
\begin{enumerate}
	\item We modified printing system call which now also traces priority of each active process. 
	\item Firstly, another field name $priority$ was added in \italb{process control block}. In our code, \ital{proc struct} was updated.
	\item The value of priority was initialised to $5$ at time of process creation.
	\item Printing was modified to take care of printing priority as well.
\end{enumerate}
\bld{Pointers on setpriority System call}
\begin{enumerate}
	\item As a continuation, we created a system call which can $setpriority$ of currently active processes.
	\item A system call was implemented which takes $process\_id$ and $priority$ as parameters.
	\item If the priority is out of range [1-20], then an error message is flagged.
	\item Otherwise, the process that matches the pid gets its priority updated.
\end{enumerate}

\section*{Part 2}
\bld{Pointers on implementing Priority Scheduler}
\begin{enumerate}
	\item This part was interesting as we replaced the default \italb{round-robin} scheduler of $xv6$ with a  \italb{priority based scheduler}.
	\item For this, the $scheduler$ method in $proc.c$ was modified.
	\item The tricky part was to ensure \ital{round-robin} property amongst the same priority processes.
	\item For this, while scheduling what was done is that we start finding the \ital{first-highest} priority process, starting with next process of currently \italb{context-switched process}.
	\item The above was done in a cyclic fashion to ensure round robin property.
	\item If there was some process with unique highest priority, it would get picked.
	\item Otherwise, it would pick up processes with same highest priority in cyclic fashion.
	\item An important point is that although this new scheduler has same \italb{space efficiency}, the time complexity has increased by an \italb{order of magnitude}.
	\item For each scheduling, we run through the entire process list and select the one with the highest priority.
	\item The above step could be efficiently implemented using a \italb{heap}, in our case a \ital{max-heap}. This would have reduced the \italb{linear factor} to \italb{logarithmic}. But since, number of processes was low, and it was more for a proof-of-concept experiment, I went with the linear search.
\end{enumerate}

\section*{Part 3}
\bld{Pointers on preventing Starvation}
\begin{enumerate}
	\item A system call was implemented which returns priority of a process, given its $process\_id$.
	\item The implementation is similar to the one in which we were setting priority of a process.
	\item The main part was incrementing priority of a process, if it has remain inactive for a long time, i.e., if the process hasn't been scheduled for long because of lower priority.
	\item For this, a variable counter was introduced, which records how many \ital{context switches} has been noticed by the process since its \ital{inception}.
	\item If this limit exceeds a threshold(50, in our case), priority of the process is increased by 1.
	\item Since processes with same priority respects \italb{round-robin}, in some way, we are ensuring $fairness$ together in our priority scheduling, which was inherent in \italb{round-robin} type scheduler.
	\item Whenever a context switch takes place, counter of all process with runnable state is updated.
\end{enumerate}

\italb{Note: Most of the scheduling works are only for Running and Runnable processes. Also, setting priority subsumes that the process is in valid state. This hasn't been explicitly handled}.

\end{document}
