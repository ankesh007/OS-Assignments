\documentclass{article}
\usepackage{graphicx}
\usepackage{geometry}
\usepackage{amsmath,mathpazo}
\usepackage{amssymb}
\usepackage{mathtools}
\usepackage{commath}
\usepackage{enumitem}
\usepackage{listings}
\usepackage{grffile}

\geometry{
  top=20mm,
}
\newcommand{\boldvec}[1]{\boldsymbol{\vec{\textbf{#1}}}}
\newcommand{\capvec}[1]{\boldsymbol{\hat{\textbf{#1}}}}
\newcommand{\bld}[1]{\textbf{#1}}
\newcommand{\ital}[1]{\textit{#1}}
\newcommand{\italb}[1]{\textbf{\textit{#1}}}
\begin{document}

\title{Operating Systems- COL331 \\Assignment 1}
\author{Ankesh Gupta\\2015CS10435}

\date{}
\maketitle

\section*{Part 1}
\bld{Pointers on System call Trace:}
\begin{enumerate}
	\item We had to trace system calls made by xv6 while its operating. 
	\item System calls in xv6 are uniquely identified by an identifier, an integer. So, to print name of the call, a mapping was maintained between unique system call id and its name. 
	\item The mapping was stored in \italb{array data structure}, which mapped system call name indexed by \ital{unique call id.}
	\item Also, as asked, another \ital{(integer)array data structure} was maintained which kept track of system call count, i.e., how many times a particular system call has been invoked since the OS(emulator) began.
	\item As soon as any system call is invoked, array's corresponding index was $incremented.$
	\item The reason for choosing array data structure was that the unique id's assigned to system calls were $contiguous$. Other data structures can also easily handle this task, but array is quite a simple and intuitive choice for the problem.
	\item The data structures are marked $static$ as they have no role outside the module.
	\item Mapping array was named $syscall\_names$. Counter array was named $call\_count$.
\end{enumerate}
\bld{Pointers on Toggling System calls}
\begin{enumerate}
	\item As a continuation, we had to create a system call which can $toggle$ tracing system calls.
	\item A $flag(variable)$ was introduced which maintained the current toggling. The variable was externed so that they are visible from external modules.
	\item Requisite changes were made in file $usys.S$ so that the system call can be recognized and correct system call id is put in \ital{eax register}.
	\item Header $syscall.h$ was also changed to allocate an unique id to new system call.
	\item Header $user.h$ was changed so that the system call is visible to user program.
	\item Actual implementation of system call was done in $sysproc.c$ to keep with convention. Here the state of toggle flag is reversed.
	\item Necessary modifications were made in $syscall.c$ so that the new system call is visible to kernel, and when called, kernel realizes what instructions to execute. 
	\item Flag was named $toggle\_flag$.
\end{enumerate}

\section*{Part 2}
\bld{Pointers on add System Call}
\begin{enumerate}
	\item In this, a system call was added which returns sum of 2 numbers.
	\item This implementation helped understand how to pass arguments to system call, and how to use them in while implementing system call $handler$.
	\item All implementation details are similar as mentioned in above case of adding sys\_toggle system call.
	\item Only difference is the way we get parameters. A call to $argint$ method was made which returns the $i^{th}$ system call argument($i$ is sent while invocation).
	\item 2 calls were made and the sum was returned, which was placed in \ital{eax register}.
	\item No extra variables or data structures were introduced. Only few look-up's were updated. 
\end{enumerate}

\section*{Part 3}
\bld{Pointers on printing currently running processes}
\begin{enumerate}
	\item We were required to list process id and process name for processes with running, runnable or sleeping state.
	\item Implementation details are again similar to above implementations.
	\item We required access to $ptable$, a structure which tracked states of all the processes. 
	\item Since it was available in $proc.c$ module, a method was declared which used it to list the required processes.
	\item The method was also externed in $sysproc.c$ so that its visible in the file and can be accessed.
	\item While printing, process's state was monitored and it was listed if: $$process->state \in \{RUNNABLE,RUNNING,SLEEPING\}$$
	\item Again, no new variables and data structures were introduced. Just few look-up's were updated.
\end{enumerate}

\end{document}
